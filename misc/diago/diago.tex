\documentclass[a4paper,12pt]{article}

\usepackage[utf8]{inputenc}
\usepackage[T1]{fontenc}
\usepackage[french]{babel}
\usepackage[top=2.5cm, left=2.5cm,right=2.5cm,bottom=2.5cm]{geometry}
\usepackage{amsmath}
\usepackage{color}


\tolerance = 3000
\setlength{\parindent}{0cm}

\title{Calcul des termes diagonaux de plusieurs estimateurs de variances usuels}
\author{Martin Chevalier (Insee, DMS) -- 12 juillet 2016}
\date{}

\begin{document}

\maketitle

Pour un échantillon $s$ et une variable $Y$ quelconque, on note $\hat{T}(Y)$ l'estimateur de son total dans la population et $\hat{V}(\hat{T}(Y))$ l'estimateur de la variance de $\hat{T}(Y)$. $\hat{V}(\hat{T}(Y))$ est une forme quadratique, c'est-à-dire qu'il existe $(q_k)_{k \in s}$ et $(q_{k,\ell})_{k \in s, \ell \in s}$ tels que: 

\begin{equation*}
\hat{V}(\hat{T}(Y)) = \sum_{k \in s}q_k y_k^2 + \sum_{k \in s} \sum_{\substack{\ell \in s \\ \ell \neq k}} q_{k,\ell}y_k y_\ell
\end{equation*}

Le terme $q_k$ est appelé terme diagonal de la forme quadratique $\hat{V}(\hat{T}(Y))$. Celui-ci est particulièrement important dans la mesure où \textbf{il intervient systématiquement dans l'application de la formule de Rao utilisée pour calculer la variance d'un sondage à plusieurs degrés (Caron, Deville, Sautory, 1998)}. 

\bigskip En effet, dans un plan de sondage à deux degrés comptant $m$ unités primaires, si on dispose d'un estimateur sans biais de la variance au premier degré $\hat{V}_{UP}(T_1(Y),...,T_m(Y))$ à partir des vrais totaux des unités primaires $T_1(Y),..., T_m(Y)$, alors on peut estimer sans biais la variance associée aux deux degrés de sondage par: $$\hat{V}(\hat{T}(Y)) = \hat{V}_{UP}(\hat{T}_1(Y),...,\hat{T}_m(Y)) + \sum_{k = 1}^{m}\left[\dfrac{1}{\pi_k^2} - q_k\right]\hat{V}_{US,k}(y_1,...,y_{n_k})$$ où $\pi_k$ est la probabilité de sélection de l'unité primaire $k$, $  \hat{V}_{US,k}$ un estimateur sans biais de la variance associée au tirage des unités secondaires au sein de l'unité primaire $k$ et $q_k$ le terme diagonal de $\hat{V}_{UP}$ associé à l'unité primaire $k$.

\bigskip En pratique, cette formule est appliquée dans l'ensemble des enquêtes réalisées par l'Insee auprès des ménages (Gros, Mousallam, 2015): d'où l'importance de connaître l'expression de ces termes diagonaux dans les cas les plus fréquents. \textbf{L'objet de cette note est de calculer les termes diagonaux de plusieurs estimateurs de variance usuels}. 

\bigskip Dans le contexte d'un plan de sondage à deux degrés, la forme quadratique à laquelle on s'intéresse est $\hat{V}_{UP}$, dans la mesure où c'est celle-ci qui détermine l'expression du terme diagonal $q_k$. Les probabilités d'inclusion simple $\pi_k$ la probabilité et doubles $\pi_{k,\ell}$ utilisées dans la suite renvoient donc dans ce contexte\footnote{La formule de Rao peut être utilisée itérativement dans le cas de plan de sondage à plus de deux degrés. On se place ici dans le cas à exactement deux degrés pour faciliter l'exposé.} au plan de sondage des unités primaires. De même, la taille (fixe) de l'échantillon d'unités tirées est notée $n$.

\bigskip \subsection*{Terme diagonal de l'estimateur de Horvitz-Thompson}

$$ \hat{V}_{HT}(\hat{T}(Y)) = \sum \limits_{k \in s} \sum \limits_{\ell \in s} \dfrac{\pi_{k,\ell} - \pi_k \pi_\ell}{\pi_{k,\ell}} \dfrac{y_k}{\pi_k}\dfrac{y_\ell}{\pi_\ell}$$

Le terme diagonal est le coefficient devant $y_k^2$, autrement dit celui qui apparaît dans la double somme pour $\ell = k$:

\begin{eqnarray*}
q_{HT,k} &=& \dfrac{\pi_{k,k} - \pi_k \pi_k}{\pi_{k,k}} \times \dfrac{1}{\pi_k^2} \\
&=& \dfrac{\pi_k - \pi_k^2}{\pi_k} \times \dfrac{1}{\pi_k^2} \\
&=& \dfrac{1}{\pi_k^2} - \dfrac{1}{\pi_k}
\end{eqnarray*}

\textbf{Remarque} La forme du terme diagonal de l'estimateur de Horvitz-Thompson conduit à une simplification de la formule de Rao:

\begin{eqnarray*}
\hat{V}(\hat{T}(Y)) &=& \hat{V}_{UP, HT}(\hat{T}_1(Y), \ldots{},\hat{T}_m(Y)) + \sum_{k = 1}^{m}\left[\dfrac{1}{\pi_k^2} - \left( \dfrac{1}{\pi_k^2} - \dfrac{1}{\pi_k} \right) \right]\hat{V}_{US,k}(y_1,...,y_{n_k}) \\
&=& \hat{V}_{UP, HT}(\hat{T}_1(Y), \ldots{},\hat{T}_m(Y)) + \sum_{k = 1}^{m}\dfrac{1}{\pi_k} \times \hat{V}_{US,k}(y_1,...,y_{n_k})
\end{eqnarray*}

Dans ce contexte, la formule de Rao coïncide avec la formule de Des Raj (Caron, Deville, Sautory, 1998). 

\bigskip \subsection*{Terme diagonal de l'estimateur de Sen-Yates-Grundy}

$$ \hat{V}_{SYG}(\hat{T}(Y)) = -\dfrac{1}{2}\sum \limits_{k \in s} \sum \limits_{\substack{\ell \in s \\ \ell \neq k}} \dfrac{\pi_{k,\ell} - \pi_k \pi_\ell}{\pi_{k,\ell}} \left(\dfrac{y_k}{\pi_k} - \dfrac{y_\ell}{\pi_\ell}\right)^2$$

\bigskip En développant le carré cette expression se réécrit:

\begin{eqnarray*}
\hat{V}_{SYG}(\hat{T}(Y)) &=& -\dfrac{1}{2} \left[ \sum \limits_{k \in s} \sum \limits_{\substack{\ell \in s \\ \ell \neq k}} \dfrac{\pi_{k,\ell} - \pi_k \pi_\ell}{\pi_{k,\ell}} \left(\dfrac{y_k}{\pi_k}\right)^2 + \sum \limits_{k \in s} \sum \limits_{\substack{\ell \in s \\ \ell \neq k}} \dfrac{\pi_{k,\ell} - \pi_k \pi_\ell}{\pi_{k,\ell}} \left(\dfrac{y_\ell}{\pi_\ell}\right)^2 \right. \\
&& \left. - 2 \sum \limits_{k \in s} \sum \limits_{\substack{\ell \in s \\ \ell \neq k}} \dfrac{\pi_{k,\ell} - \pi_k \pi_\ell}{\pi_{k,\ell}} \left( \dfrac{y_k}{\pi_k} \dfrac{y_\ell}{\pi_\ell}\right) \right]
\end{eqnarray*}

\bigskip En inversant par symétrie $k$ et $\ell$ dans le deuxième terme, cette expression se simplifie:

\begin{eqnarray*}
\hat{V}_{SYG}(\hat{T}(Y)) &=& - \sum \limits_{k \in s} \sum \limits_{\substack{\ell \in s \\ \ell \neq k}} \dfrac{\pi_{k,\ell} - \pi_k \pi_\ell}{\pi_{k,\ell}} \left(\dfrac{y_k}{\pi_k}\right)^2 + \sum \limits_{k \in s} \sum \limits_{\substack{\ell \in s \\ \ell \neq k}} \dfrac{\pi_{k,\ell} - \pi_k \pi_\ell}{\pi_{k,\ell}} \left( \dfrac{y_k}{\pi_k} \dfrac{y_\ell}{\pi_\ell}\right)
\end{eqnarray*}

\bigskip Il ne reste alors plus qu'à chercher la valeur du terme diagonal $q_{SYG,k}$, autrement dit la valeur du coefficient devant $y_k^2$ quand toutes les expressions sont développées au maximum:

\begin{eqnarray*}
q_{SYG,k} &=& - \dfrac{1}{\pi_k^2}\sum \limits_{\substack{\ell \in s \\ \textcolor{red}{\ell \neq k}}} \dfrac{\pi_{k,\ell} - \pi_k \pi_\ell}{\pi_{k,\ell}}
\end{eqnarray*}

\bigskip \subsection*{Terme diagonal de l'estimateur de Deville}

\bigskip Pour une variable $Y$ quelconque, un estimateur de la variance de l'estimateur du total de $Y$ peut également être obtenu en utilisant l'approximation de Deville: $$ \hat{V}_D(\hat{T}(Y)) = \dfrac{n}{n-1}\sum_{k \in s} ( 1 - \pi_k)\left[\dfrac{y_k}{\pi_k} - \dfrac{\sum_{\ell \in s} (1 - \pi_\ell) \frac{y_\ell}{\pi_\ell}}{\sum_{\ell \in s} (1 - \pi_\ell)}\right]^2 $$

À noter que plusieurs autres formes proches sont qualifiées d'approximations ou formules de Deville (Caron, Deville, Sautory, 1998, p.~8). 

\bigskip En développant le carré cette expression se réécrit:

\begin{eqnarray*}
\hat{V}_D(\hat{T}(Y)) &=& \dfrac{n}{n-1}\left[\sum_{k \in s} ( 1 - \pi_k)\left(\dfrac{y_k}{\pi_k}\right)^2
+ \sum_{k \in s} ( 1 - \pi_k) \left(\dfrac{\sum_{\ell \in s} (1 - \pi_\ell) \frac{y_\ell}{\pi_\ell}}{\sum_{\ell \in s} (1 - \pi_\ell)} \right)^2 \right. \\
&& \left. - 2 \sum_{k \in s} ( 1 - \pi_k) \dfrac{y_k}{\pi_k} \times \dfrac{\sum_{\ell \in s} (1 - \pi_\ell) \frac{y_\ell}{\pi_\ell}}{\sum_{\ell \in s} (1 - \pi_\ell)} \right]
\end{eqnarray*}

En simplifiant les deuxième et troisième termes, on obtient alors:
\begin{eqnarray*}
\hat{V}_D(\hat{T}(Y)) &=& \dfrac{n}{n-1}\left[\sum_{k \in s} ( 1 - \pi_k)\left(\dfrac{y_k}{\pi_k}\right)^2
- \dfrac{\left(\sum_{k \in s} (1 - \pi_k) \frac{y_k}{\pi_k}\right)^2}{\sum_{k \in s} (1 - \pi_k)} \right]
\end{eqnarray*}

Il ne reste alors plus qu'à chercher la valeur du terme diagonal $q_{D,k}$, autrement dit la valeur du coefficient devant $y_k^2$ quand toutes les expressions sont développées au maximum:
\begin{eqnarray*}
q_{D,k} &=& \dfrac{n}{n-1}\left[(1 - \pi_k)\dfrac{1}{\pi_k^2}
- \dfrac{(1 - \pi_k)^2 \frac{1}{\pi_k^2}}{\sum_{\ell \in s} (1 - \pi_\ell)} \right] \\
&=& \dfrac{n}{n-1} \times (1 - \pi_k) \times \dfrac{1}{\pi_k^2} \times \left[1 - \dfrac{(1 - \pi_k)}{\sum_{\ell \in s} (1 - \pi_\ell)} \right]
\end{eqnarray*}

\bigskip \subsection*{Terme diagonal de l'estimateur de Wolter en cas de \textit{collapse} de strates}

Dans le cadre d'un \textbf{sondage aléatoire simple stratifié}, le nombre d'unités tirées par strate peut être faible. Dans ce contexte, l'estimateur de variance de Deville peut être peu performant (Caron, Deville, Sautory, 1998,p.~8). Plus encore, si moins de 2 unités sont échantillonnées dans certaines strates alors aucun estimateur direct de la variance n'est calculable. 

\bigskip Dans cette configuration, il est fréquent de fusionner les strates problématiques par groupes de 2 ou 3 selon un certain critère\footnote{Le choix du critère n'est pas examiné ici. Pour un début de discussion, voir (Wolter, 2008, p.~50 \textit{sqq}).}: on parle de \textit{collapse} de strates. On part de $H$ strates indicées par $h$ pour aboutir à $P$ pseudo-strates indicées par $p$. $N_h$,$n_h$,$N_p$,$n_p$ désignent respectivement le nombre d'unités appartenant à la strate $h$ dans la population et dans l'échantillon, le nombre d'unités appartenant à la pseudo-strate $p$ dans la population et dans l'échantillon. On note de plus $H_p$ le nombre de strates dans la pseudo-strate $p$. 

\bigskip On note enfin $\hat{T}_h(Y)$ l'estimateur du total de $Y$ dans la strate $h$ et $\hat{T}_p(Y)$ l'estimateur du total de $y$ dans la pseudo-strate $p$. Comme on a affaire à des sondages aléatoires simples au sein de chaque strate, l'estimateur d'Horvitz-Thompson vaut: $$ \hat{T}_h(Y) = \frac{N_h}{n_h} \sum_{k \in s_h} y_k \quad \text{et} \quad \hat{T}_p(Y) = \sum_{h = 1}^{H_p} \frac{N_h}{n_h} \sum_{k \in s_h} y_k $$

\bigskip Sous l'hypothèse que chaque pseudo-strate est constituée d'au moins 2 strates ($\forall k \quad H_k > 1$), l'estimateur de variance suivant (Wolter, 2008, p.~53) a la propriété de majorer la variance de l'estimateur sous le vrai plan de sondage (tirage de 1 unité dans certaines strates): $$ V_C(\hat{T}(Y)) = \sum_{p = 1}^{P} \dfrac{H_p}{H_p - 1} \sum_{h = 1}^{H_p} \left[\hat{T}_h(Y) - \dfrac{\hat{T}_p(Y)}{H_p} \right]^2$$

\bigskip En développant le carré cette expression se réécrit:

\begin{eqnarray*}
V_C(\hat{T}(Y)) &=& \sum_{p = 1}^{P} \dfrac{H_p}{H_p - 1} \left[ \sum_{h = 1}^{H_p} \hat{T}_h(Y)^2 +\sum_{h = 1}^{H_p} \left(\dfrac{\hat{T}_p(Y)}{H_p}\right)^2 - 2\left(\sum_{h = 1}^{H_p} \hat{T}_h(Y)\right) \dfrac{\hat{T}_p(Y)}{H_p}\right] \\
&=& \sum_{p = 1}^{P} \dfrac{H_p}{H_p - 1} \left[ \sum_{h = 1}^{H_p} \hat{T}_h(Y)^2 + H_p\times \left(\dfrac{\hat{T}_p(Y)}{H_p}\right)^2 - 2 \dfrac{\hat{T}_p(Y)^2}{H_p}\right] \\
&=& \sum_{p = 1}^{P} \dfrac{H_p}{H_p - 1} \left[ \sum_{h = 1}^{H_p} \hat{T}_h(Y)^2 - \dfrac{\hat{T}_p(Y)^2}{H_p}\right] \\
&=& \sum_{p = 1}^{P} \dfrac{H_p}{H_p - 1} \left[ \sum_{h = 1}^{H_p} \left(\frac{N_h}{n_h} \sum_{k \in s_h} y_k\right)^2 - \dfrac{\left(\sum_{h = 1}^{H_p} \frac{N_h}{n_h} \sum_{k \in s_h} y_k \right)^2}{H_p}\right] \\
\end{eqnarray*}

\bigskip Il ne reste alors plus qu'à chercher la valeur du terme diagonal $q_{C,k}$, autrement dit la valeur du coefficient devant $y_k^2$ quand toutes les expressions sont développées au maximum:
\begin{eqnarray*}
q_{C,k} &=& \dfrac{H_p}{H_p - 1} \left[\left(\dfrac{N_h}{n_h}\right)^2 - \left(\dfrac{N_h}{n_h}\right)^2 \times \dfrac{1}{H_p}\right] = \left(\dfrac{N_h}{n_h}\right)^2 \dfrac{H_p}{H_p - 1} \times \left(1 - \dfrac{1}{H_p}\right) = \left(\dfrac{N_h}{n_h}\right)^2
\end{eqnarray*}

avec $h$ et $p$ respectivement la strate et la pseudo-strate auxquelles appartient l'unité $k$.

\bigskip \subsection*{Références}

\bigskip \textsc{Caron N., Deville J.-C., Sautory O.} (1998) \textit{Estimation de précision de données issues d'enquêtes : document méthodologique sur le logiciel POULPE}, Document de travail de l'Insee n 9806, 44~p.

\bigskip \textsc{Gros E., Moussallam K.} (2015) \textit{Les méthodes d'estimation de la précision pour les enquêtes ménages de l'Insee tirées dans Octopusse}, Document de travail de l'Insee M~2015/03, 96~p.

\bigskip \textsc{Wolter K.} (2008) \textit{Introduction to Variance Estimation}, Springer, 450~p.

\end{document}